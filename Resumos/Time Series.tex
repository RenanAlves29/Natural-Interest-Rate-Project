\chapter{Modelos com Time Series}
%
%
\section{\citet{DelNegro:2019}: Global trends in interest rates}

Para responder a essas questões, estudamos a dinâmica conjunta das taxas de juros de curto e longo prazo, inflação e consumo para sete economias agora avançadas desde 1870. Fazemos isso por meio de um modelo de série temporal flexível - um vetor de autorregressão (VAR) com tendências. Essa ferramenta econométrica nos permite usar a teoria econômica para modelar e interpretar as relações de longo prazo entre as variáveis, enquanto permanecemos agnósticos sobre se essas restrições se mantêm em outras frequências. Por exemplo, a ausência de oportunidades de arbitragem no longo prazo implica que podemos interpretar a tendência comum estimada das taxas de juros reais entre os países como a tendência da taxa de juros real mundial. O mesmo referencial teórico também sugere uma decomposição dessa tendência em alguns de seus fatores potenciais, como o crescimento do consumo global.

As taxas de juros em nosso conjunto de dados referem-se a títulos do governo ou substitutos próximos, que são relativamente seguros e líquidos em comparação com outros ativos de emissão privada. Portanto, permitimos que o rendimento de conveniência para a segurança e a liquidez oferecidas por esses ativos “seguros” desempenhem um papel na condução da seção internacional de retornos. Para medir esse rendimento de conveniência, a análise empírica também inclui o rendimento de títulos corporativos Baa da Moody's para os Estados Unidos.

Quatro resultados principais emergem de nossa análise empírica. Em primeiro lugar, a tendência estimada da taxa de juros real mundial manteve-se estável em torno de valores um pouco abaixo de $2\%$ ao longo da década de 1940. Aumentou gradualmente após a Segunda Guerra Mundial para um pico de cerca de $3\%$ no final dos anos setenta, mas tem diminuído desde então, caindo para quase zero em 2016, o último ano de dados disponível. O nível exato dessa tendência é cercado por uma incerteza substancial, mas a queda nas últimas décadas é estimada com precisão. Um declínio dessa magnitude não tem precedentes em nossa amostra. Nem mesmo ocorreu durante a Grande Depressão na década de 1930.

Em segundo lugar, a tendência da taxa de juros mundial desde o final da década de 1970 coincide essencialmente com a dos EUA. Em outras palavras, a tendência dos EUA é a tendência global nas últimas quatro décadas. Na verdade, esse tem sido cada vez mais o caso para quase todos os outros países em nossa amostra: as tendências idiossincráticas estão desaparecendo desde o final dos anos 1970. Essa convergência nas taxas de juros entre os países é indiscutivelmente o resultado da crescente integração nos mercados internacionais de ativos.

Terceiro, a tendência de declínio da taxa de juros real mundial nas últimas décadas é impulsionada em extensão significativa por um aumento nos convenience yield, que aponta para um desequilíbrio crescente entre a demanda global por segurança e liquidez e sua oferta. Essa contribuição está especialmente concentrada no período desde meados da década de 1990, apoiando a visão de que a crise financeira asiática de 1997 e a inadimplência da Rússia em 1998, com o subsequente colapso do LTCM, foram pontos de inflexão importantes no surgimento de desequilíbrios globais.

Quarto, um declínio global na taxa de crescimento do consumo per capita, possivelmente vinculado a mudanças demográficas, é outro fator notável que está empurrando as taxas reais globais para baixo. Sua contribuição é comparável em magnitude à do convenience yield desde 1980, mas apenas cerca de metade da importância nos últimos vinte anos (e estimada com menos precisão).

Uma implicação importante dessas descobertas é que o persistente
os ventos contrários macroeconômicos decorrentes da crise financeira, incluindo os efeitos das políticas extraordinárias que foram postas em prática para combatê-la, estão longe de ser a única causa do ambiente de juros baixos. Forças seculares de longa data conectadas com um declínio no crescimento econômico desde o início dos anos 1980 e o aumento dos convenience yield desde o final dos anos 1990 também parecem ser os culpados cruciais, embora essas tendências possam ter sido exacerbadas pela crise.

Nossos resultados estão intimamente alinhados com esta literatura ao enfatizar o importante papel dos convenience yield na geração de retornos internacionais. Eles são complementares porque nos concentramos na contribuição desses fatores para impulsionar o comovimento internacional das taxas de juros em baixas frequências e por um período de tempo muito mais longo. Portanto, podemos abordar explicitamente a questão de como a dimensão global da demanda por segurança e liquidez moldou o declínio secular das taxas reais em todo o mundo nas últimas décadas, o que também nos permite fazer contato com a vasta literatura sobre r * discutido acima.

Dez anos após a fase mais aguda da crise financeira global, as taxas de juros permanecem em níveis historicamente baixos ou próximos a muitos para muitos países. Estudamos os fatores seculares desse ambiente de taxas de juros baixas através das lentes de uma autorregressão vetorial com tendências comuns, usando dados históricos de sete países a partir de 1870. Descobrimos que a tendência da taxa de juros real mundial segura, que era praticamente estável em um pouco abaixo de $2\%$ por mais de cem anos, caiu significativamente nas últimas três décadas. Esta tendência global, que identificamos como o componente comum nos movimentos de baixa frequência dos rendimentos reais sobre ativos seguros e líquidos (títulos do governo ou substitutos próximos) nas sete economias em nossa amostra, se assemelha muito às tendências para todas as economias avançadas, incluindo os Estados Unidos, no período recente. Descobrimos que as tendências específicas de cada país praticamente desapareceram desde os anos 1970.

Esse declínio secular nas taxas reais globais é impulsionado principalmente por um aumento no prêmio que os investidores internacionais estão dispostos a pagar para manter ativos seguros e líquidos, bem como pelo menor crescimento econômico em todo o mundo. A última tendência tem exercido pressão descendente sobre as taxas reais desde cerca de 1980, possivelmente ligada a mudanças demográficas, enquanto a primeira surgiu no final da década de 1990. Este momento aponta para a escassez de ativos seguros no contexto de um excesso de poupança global como uma força secular fundamental por trás do ambiente de taxas de juros baixas.
%
%
\section{\citet{Johannsen:2018}: A time series model of interest rates with the effective lower bound}
Este artigo modela as taxas de juros nominais, juntamente com outros dados macroeconômicos, usando um modelo de série temporal flexível que incorpora explicitamente o limite inferior efetivo (ELB) nas taxas de juros nominais. Empregamos um dispositivo de modelagem a que nos referimos como “shadow rate” - a taxa de juros nominal que prevaleceria na ausência do ELB.

Usamos nossa abordagem para estimar um modelo de ciclo de tendência de dados dos EUA sobre taxas de juros, atividade econômica e inflação em uma amostra que inclui o período recente no ELB. Desde a crise financeira global de 2008, as taxas de juros reais têm estado historicamente baixas, o que levou alguns a argumentar que o nível de longo prazo da taxa de juros real caiu. Incorporamos a estimativa das taxas reais de tendência em um modelo de volatilidade estocástica cujas estimativas veem quedas recentes nas taxas reais como amplamente cíclicas
na natureza.

Como resultado, nossas estimativas da tendência da taxa real são menos variáveis do que as relatadas. Encontramos uma incerteza considerável em torno das estimativas da taxa real no longo prazo. À luz das amplas faixas de incerteza em torno das estimativas de tendência, a modesta tendência de queda em nossas estimativas de tendência da taxa real não é significativa.Como resultado, nossas estimativas da tendência da taxa real são menos variáveis do que as relatadas. Encontramos uma incerteza considerável em torno das estimativas da taxa real no longo prazo. À luz das amplas faixas de incerteza em torno das estimativas de tendência, a modesta tendência de queda em nossas estimativas de tendência da taxa real não é significativa.

Modelar explicitamente o ELB tem grandes efeitos na inferência sobre taxas de juros esperadas fora da amostra nos últimos anos. Nossas shadow rate estimadas são menores do que o ELB por definição, e nosso modelo oferece caminhos previstos para as taxas de juros de curto prazo futuras que incluem períodos prolongados no ELB. Comparamos as previsões de taxas de juros de nosso modelo com as previsões da Survey of Professional Forecasters (SPF) e descobrimos que nosso modelo tem um desempenho melhor do que o SPF em horizontes mais longos.

No ELB, a shadow rate é uma variável de estado não observada que importa para a previsão de resultados futuros na taxa de juros e outras variáveis. Quando a taxa de referência está acima do ELB, a shadow rate é igual à taxa de referência e, portanto, também observada. Em ambos os casos, as inovações na shadow rate podem ser interpretadas como refletindo mudanças na política monetária - sejam elas implementadas na forma de variações convencionais na taxa de juros quando o ELB não é vinculativo, ou por meio de ferramentas não convencionais (como compras de ativos ou a termo orientação) caso contrário. Nesse espírito, estimamos as respostas de impulso a choques de política monetária, que são identificados pela imposição de restrições de curto prazo às surpresas da shadow rate.

Além disso, nosso modelo de volatilidade estocástica gera respostas de impulso que variam no tempo: Nossos choques de política são identificados a partir de restrições convencionais de curto prazo impostas à estrutura de covariância dos erros de previsão de forma reduzida do modelo. À medida que a importância relativa dos choques nas tendências e nos ciclos muda, também muda a estrutura de covariância dos erros de previsão de forma reduzida do modelo, bem como a persistência de seus efeitos.

Descobrimos que os choques de política monetária identificados a partir de inovações na taxa de sombra afetam os spreads de rendimento de forma mais marcante quando o ELB é vinculativo, consistente com a noção de que as taxas de sombra capturam os efeitos de políticas não convencionais. É importante ressaltar que nossos resultados sugerem que uma acomodação monetária adicional durante o período mais profundo da recessão mais recente teria fornecido mais estímulo do que a acomodação monetária fornecida em outros momentos.

Neste artigo, desenvolvemos uma metodologia para contabilizar o ELB em modelos de séries temporais de taxas de juros nominais. Nosso método torna os modelos lineares gaussianos acessíveis ao ELB, mas também pode ser aplicado a modelos de parâmetros variáveis no tempo que são apenas condicionalmente lineares. Por exemplo, nossa aplicação empírica é baseada em um modelo de componentes não observados com volatilidade estocástica. Demonstramos como estimar os parâmetros e estados latentes de tal modelo com um amostrador MCMC Bayesiano padrão.

A consideração adequada do ELB tem, evidentemente, efeitos drásticos nas previsões das taxas de juros. Mesmo entre os preditores que aderem ao ELB, nossa abordagem de shadow rate se sai bem em comparação com vários concorrentes. Também estimamos uma tendência comum das taxas reais, definida como uma previsão de longo prazo da taxa de juros real, e não encontramos nenhum declínio significativo na tendência da taxa real desde os anos 1990.

Finalmente, nosso modelo gera respostas de impulso que variam no tempo a choques de política monetária. Descobrimos que os choques de política monetária identificados a partir de inovações na taxa de sombra afetam os spreads de rendimento de forma mais marcante quando o ELB é vinculativo, consistente com a noção de que as taxas de sombra capturam os efeitos de políticas não convencionais. É importante ressaltar que nossos resultados sugerem que a acomodação monetária durante as profundezas da recessão mais recente teria fornecido mais estímulo do que em outros momentos.
%
%
\section{\citet{Hamilton:2016}: The Equilibrium Real Funds Rate: Past, Present, and Future}

Examinamos o comportamento, os determinantes e as implicações do nível de equilíbrio da taxa real de fundos federais, definida como a taxa consistente com o pleno emprego e inflação estável no médio
prazo. Tiramos três conclusões principais. Em primeiro lugar, a incerteza em torno da taxa de equilíbrio é grande e sua relação com a tendência de crescimento do PIB muito mais tênue do que se acredita. Nossa análise narrativa e econométrica usando dados de vários países e remontando ao século 19 suporta uma ampla gama de estimativas centrais plausíveis para o nível atual da taxa de equilíbrio, de um pouco mais de $0\%$ ao consenso pré-crise de $2\%$. Em segundo lugar, apesar dessa incerteza, somos céticos em relação à visão da “estagnação secular” de que a taxa de equilíbrio permanecerá próxima de zero por muitos anos. As evidências de estagnação secular antes da crise de 2008 são fracas, e a decepcionante recuperação pós-2008 é melhor explicada por ventos contrários prolongados, mas em última análise temporários, causados pelo excesso de oferta de moradias, desalavancagem das famílias e dos bancos e contenção fiscal. Uma vez que esses ventos contrários diminuíram no início de 2014, o crescimento dos EUA de fato acelerou a um ritmo bem acima do potencial. Em terceiro lugar, a incerteza em torno da taxa de equilíbrio implica que uma regra de política monetária com mais inércia do que implícita nas versões padrão da regra de Taylor poderia estar associada a menores desvios do produto e da inflação dos objetivos do Fed. Nossas simulações usando o modelo FRB / US da equipe do Fed mostram que o reconhecimento explícito dessa incerteza resulta em um caminho de normalização posterior, mas mais íngreme para a taxa de fundos em comparação com o "ponto" mediano no Resumo de Projeções Econômicas do FOMC.

Neste artigo, abordamos a questão de um “novo neutro” examinando a experiência de um grande número de países, embora com foco nos EUA. descrevemos os dados e procedimentos que usaremos para construir as taxas reais ex-ante usadas em nossa análise. Esses dados remontam a dois séculos para alguns países e também incluem dados mais detalhados sobre a experiência mais recente das economias da OCDE. Também observamos a estratégia que freqüentemente usamos para fazer afirmações empíricas sobre a taxa de equilíbrio: na maior parte, olharemos para as médias ou médias móveis de nossas medidas de taxas reais; em nenhum momento iremos estimar um modelo estrutural.

Consideramos as médias móveis como medidas (ruidosas) da taxa de equilíbrio e da taxa de crescimento da tendência. Usando observações de longas séries de tempo para os Estados Unidos e também a experiência dos países da OCDE desde 1970, investigamos a relação entre taxas reais seguras e tendência de crescimento do produto. Descobrimos algumas evidências de que taxas de crescimento de tendência mais altas estão associadas a taxas reais médias mais altas. No entanto, essa descoberta é sensível à amostra particular de dados que é usada. E mesmo para as amostras com relação positiva, a correlação entre o crescimento e as taxas médias é modesta. Concluímos que fatores além das mudanças na tendência da taxa de crescimento são centrais para explicar por que a taxa real de equilíbrio muda ao longo do tempo.

Fornecemos uma história narrativa dos determinantes da taxa real nos EUA tentando identificar os principais fatores que podem ter movido a taxa de equilíbrio ao longo do tempo. Concluímos que as mudanças ao longo do tempo nas taxas de desconto pessoal, regulamentação financeira, tendências na inflação, bolhas e ventos contrários cíclicos tiveram efeitos importantes sobre a taxa real observada em média ao longo de qualquer década. Discutimos a hipótese da estagnação secular em detalhes. No geral, consideramos isso pouco convincente, argumentando que provavelmente confunde uma recuperação atrasada com uma demanda agregada cronicamente fraca. Nossa análise sugere que o ciclo atual pode ser semelhante aos dois últimos, com uma “normalização” tardia tanto da economia quanto da taxa de fundos. Nossa abordagem narrativa sugere que a taxa de equilíbrio pode ter caído, mas provavelmente apenas ligeiramente. O crescimento da tendência presumivelmente menor implica uma taxa de equilíbrio abaixo da média de $2\%$ que prevaleceu recentemente, talvez em algum lugar na faixa de $1\%$ a $2\%$.

Realizamos algumas análises estatísticas dos dados de longo prazo dos EUA e descobrimos, de acordo com nossa história narrativa, bem como com resultados empíricos encontrados por outros pesquisadores em conjuntos de dados do pós-guerra, que nós pode rejeitar a hipótese de que a taxa de juros real converge ao longo do tempo para alguma constante fixa. Encontramos uma relação que parece estável. A taxa real dos EUA é cointegrada com uma medida que é semelhante à mediana de uma média de 30 anos das taxas reais em todo o mundo. Quando a taxa dos EUA está abaixo da taxa mundial de longo prazo (como no início de 2015), podemos ter alguma confiança de que a taxa dos EUA vai subir, consistente com a conclusão de nossa análise narrativa. O modelo prevê que a taxa real de longo prazo dos EUA e do mundo se estabilize em um valor em torno de meio por cento em cerca de três anos. No entanto, como a própria taxa mundial também é não estacionária, sem tendência clara de reverter para uma média fixa, a incerteza associada a essa previsão torna-se maior à medida que tentamos olhar para o futuro.
%
%
\section{\citet{Rudebusch:2019}: A New Normal for Interest Rates? Evidence from Inflation-Indexed Debt}

Dadas essas armadilhas potenciais de uma estimativa baseada em macro, nos voltamos para modelos financeiros e dados para fornecer uma abordagem alternativa para estimar a taxa de juros real de equilíbrio. Usamos os preços da dívida indexada à inflação, a saber, U.S. Treasury Inflation-Protected Securities (TIPS). Esses títulos têm pagamentos de cupom e principal que se ajustam às variações do Índice de Preços ao Consumidor (IPC) e, assim, compensam os investidores pela erosão de poder compra devido à inflação. Os preços desses títulos podem fornecer uma leitura bastante direta dos rendimentos reais desde 1997, quando o programa TIPS foi lançado. Assumimos que as expectativas de longo prazo embutidas nos preços de TIPS refletem as visões dos participantes do mercado financeiro sobre o estado estacionário da economia, incluindo a taxa de juros de equilíbrio. Nossa medida da taxa natural baseada em finanças tem várias vantagens potenciais em relação às estimativas baseadas em macro. Mais notavelmente, nossa medida da taxa de equilíbrio não depende da obtenção de uma especificação correta da dinâmica do produto e da inflação - ao contrário de estimativas anteriores que dependem de uma representação macroeconômica específica. Além disso, nossa medida pode ser obtida em tempo real na mesma alta frequência que os dados do preço do título subjacente e é baseada em dados do mercado financeiro e, portanto, é naturalmente foward-looking.

Ainda assim, o uso de TIPS para medir a taxa de juros real de curto prazo em estado estacionário apresenta seus próprios desafios empíricos. Uma dificuldade é que os preços dos títulos indexados à inflação incluem um prêmio de prazo real. Dada a inclinação geralmente ascendente da curva de rendimento TIPS, o prêmio de prazo real parece positivo em média, mas sua variabilidade é desconhecida. Além disso, apesar do valor nocional bastante elevado de TIPS em circulação, esses títulos provavelmente enfrentam um risco de liquidez considerável.

Para estimar a taxa de juros de equilíbrio na presença de liquidez e prêmios de prazo real, usamos modelos de estrutura a termo dinâmica livre de arbitragem de rendimentos reais. A formulação teórica livre de arbitragem do modelo fornece a identificação de um prêmio de prazo real variável no preço de TIPS. Além disso, nosso modelo é estimado usando os preços de títulos individuais, em vez da entrada mais comum de rendimentos de curvas sintéticas ajustadas

Para robustez, consideramos dois modelos dinâmicos de estrutura a termo diferentes. Um é mais padrão, sem tratamento explícito separado do prêmio de liquidez - riscos de prazo e de liquidez são modelados implicitamente juntos. O segundo modelo é ampliado com um fator de risco de liquidez explícito. Este modelo identifica um fator de liquidez geral de TIPS e a carga de cada título individual nesse fator a partir da seção transversal dos preços de TIPS ao longo do tempo - com cada título possuindo um tempo diferente desde a emissão e até o vencimento.

Usando modelos e dados macroeconômicos, muitos pesquisadores investigaram a contribuição para a tendência de baixa dos rendimentos nas últimas décadas a partir de uma taxa de juros real de equilíbrio em queda. No entanto, a incerteza sobre a especificação macroeconômica correta levou alguns a questionar a validade das estimativas macroeconômicas resultantes da taxa natural. Evitamos esse debate introduzindo uma medida da taxa real de equilíbrio baseada em finanças, obtida exclusivamente de modelos dinâmicos de estrutura a termo estimados com base nos preços de títulos indexados à inflação. Ajustando os prêmios de liquidez do TIPS e os prêmios de prazo real, descobrimos as expectativas dos investidores quanto à taxa de juros curta real sem atrito subjacente para o período de cinco anos, começando cinco anos à frente. Nossa medida resultante da taxa natural de juros exibe um declínio gradual nas últimas duas décadas para um nível essencialmente zero. Além disso, as projeções do modelo sugerem que a taxa natural provavelmente permanecerá bastante baixa por algum tempo.

Vemos nossa análise de taxa de equilíbrio baseada em finanças como um complemento às anteriores macrobaseadas. Como as estimativas baseadas em macro, uma estimativa baseada em finanças também está sujeita a críticas sobre a especificação do modelo e o conteúdo de informação dos dados disponíveis. À luz de tais críticas, a faixa de incerteza associada a uma estimativa baseada em finanças não parece ser necessariamente menor do que aquela em torno das estimativas baseadas em macro. No entanto, os modelos e dados subjacentes nas duas abordagens são tão diferentes que os intervalos de confiança provavelmente também não estão correlacionados, o que sugere um valor substancial da construção e comparação de estimativas baseadas em finanças e macro. É claro que uma abordagem conjunta que combine dados macroeconômicos e do mercado financeiro parece ser particularmente promissora para pesquisas futuras. Na verdade, nossa medida poderia ser incorporada a uma análise macroeconômica e financeira conjunta expandida - particularmente com o objetivo de compreender melhor os determinantes da nova normalidade mais baixa para as taxas de juros.
%
%
\section{\citet{Lubik:2015}: Calculating the Natural Rate of Interest: A Comparison of Two Alternative Approaches}

Um modelo autoregressivo de vetor de parâmetro variável no tempo (TVPVAR) é uma estrutura flexível para estudar as relações complexas entre os dados macroeconômicos. É um modelo de série temporal que explica a evolução das variáveis econômicas em função de seus próprios valores defasados e choques aleatórios. O que distingue um TVPVAR da abordagem VAR mais padrão é que os parâmetros do modelo, nomeadamente os coeficientes de defasagem e as variâncias dos choques económicos, podem variar ao longo do tempo. Esta estrutura é, portanto, capaz de capturar uma variedade de comportamentos não lineares que são aparentes em séries temporais macroeconômicas, como movimentos assimétricos de variáveis ao longo do ciclo de negócios ou o declínio geral da volatilidade macroeconômica desde meados da década de 1980.

Esta última questão é especialmente pertinente à questão do comportamento da taxa natural de juros. Como mostram os números acima, parece haver um declínio secular na taxa real e sua contraparte natural, conforme estimado por Laubach e Williams, bem como mudanças na volatilidade do primeiro durante o período da amostra. Além disso, a existência do limite inferior zero na taxa de juros nominal introduz, por si só, não linearidade nas relações macroeconômicas. Essas considerações, portanto, tornam um TVP-VAR uma estrutura atraente para capturar a taxa natural.

O que distingue esta abordagem do método de Laubach e Williams é que o TVP-VAR impõe muito menos de uma estrutura econômica. Enquanto Laubach e Williams postularam relações econômicas entre as principais variáveis macroeconômicas que podem ou não ser suportadas, um TVP-VAR é amplamente agnóstico nessa dimensão. Ele simplesmente captura o co-movimento entre essas variáveis de uma maneira flexível. A identificação da taxa natural em Laubach e Williams repousa criticamente nas suposições que regem
modelo econômico subjacente.

Estimamos um simples TVP-VAR para três variáveis— a taxa de crescimento do produto interno bruto real, a taxa de inflação PCE e a medida da taxa de juros real usada por Laubach e Williams - durante o período de amostra de 1961 até o segundo trimestre de 2015. Conforme discutido acima, a abordagem TVP-VAR é bem adequado para capturar as características seculares e de ciclo de negócios desse intervalo de dados. Propomos como medida da taxa natural de juros a previsão condicional de longo prazo da taxa real observada. Nosso horizonte de tempo escolhido é de cinco anos, e a previsão é calculada para cada ponto de dados desde 1967. Em contraste com os VARs de coeficiente fixo estacionário, as previsões não revertem para a média da amostra porque os coeficientes do TVP-VAR seguem caminhos aleatórios.