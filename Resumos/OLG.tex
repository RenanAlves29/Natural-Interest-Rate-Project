\chapter{Modelos OLG}

\section{\citet{Ferrero:2016}: Demographics and real interest rates: Inspecting the mechanism}

As taxas de juros reais estão tendendo para baixo há mais de duas décadas em muitos países. Esses movimentos de baixa frequência sugerem que outras forças além das políticas monetárias acomodativas devem estar em jogo.

As tendências demográficas são uma explicação candidata natural para taxas de juros reais baixas e em declínio. O mundo está passando por uma dramática transição demográfica. Na maioria das economias avançadas, as pessoas tendem a viver mais tempo. No Japão, nos EUA e na Europa Ocidental, a expectativa de vida ao nascer aumentou cerca de 10 anos entre 1960 e 2010, e as novas gerações continuaram a esperar que a longevidade aumentasse. Ao mesmo tempo, apesar da imigração, as taxas de crescimento populacional estão diminuindo a um ritmo acelerado e, em alguns casos (por exemplo, no Japão), tornam-se negativas. A combinação da desaceleração do crescimento populacional e o aumento da longevidade implica um aumento notável na taxa de dependência - ou seja, a razão entre pessoas com 65 anos ou mais e pessoas de 15 a 64 anos. As conseqüências dessa transição demográfica são de grande alcance e têm importantes implicações macroeconômicas.

Neste artigo, enfocamos as conseqüências da transição demográfica para as taxas de juros reais. Ilustramos três canais através dos quais essa transição pode afetar a taxa de juros real de equilíbrio usando um modelo de ciclo de vida tratável. Para uma determinada idade de aposentadoria, um aumento na expectativa de vida aumenta o período de aposentadoria e gera incentivos adicionais para economizar durante todo o ciclo de vida. Este efeito tende a ser mais forte se os agentes acreditarem que os sistemas públicos de previdência não serão capazes de arcar com a carga adicional gerada pelo envelhecimento da população. Portanto, um aumento na longevidade - e suas expectativas - tende a pressionar a taxa de juros real, à medida que os agentes acumulam suas poupanças em antecipação a um período de aposentadoria mais longo.

Uma queda na taxa de crescimento da população produz dois efeitos opostos nas taxas de juros reais. Por um lado, o menor crescimento populacional leva a uma maior relação capital-trabalho, o que deprime o produto marginal do capital. Esse “efeito de oferta” é muito semelhante a uma desaceleração permanente no crescimento da produtividade, reduzindo as taxas de juros reais. Por outro lado, no entanto, o menor crescimento populacional acaba aumentando a taxa de dependência. Como os aposentados têm uma menor propensão marginal a poupar, essa mudança na composição da população é semelhante a um “efeito de demanda” que eleva o consumo agregado e pressiona para cima as taxas de juros reais de equilíbrio.

Calibramos o modelo para capturar características marcantes da transição demográfica em economias desenvolvidas e quantificamos os efeitos dos canais acima mencionados. O efeito geral de uma transição demográfica prototípica é reduzir a taxa de juros de equilíbrio em uma quantidade significativa. Em particular, para nosso “país representativo desenvolvido”, a taxa real anual de equilíbrio cai 1,5 pontos percentuais entre 1990 e 2014. O aumento na expectativa de vida é responsável pela maior parte da queda na taxa de juros real.
%
%
\section{\citet{Kara:2016}: Interest Rate Effects of Demographic Changes in a New Keynesian Life-Cycle Framework}
Este artigo parte da observação de que a maioria dos países industrializados está sujeita a mudanças demográficas de longa duração. Duas características principais dessas mudanças, que são particularmente pronunciadas em vários países europeus, são uma desaceleração secular no crescimento populacional e um aumento substancial na longevidade.

Este artigo tem o objetivo de desenvolver uma estrutura de economia fechada para a análise da política monetária que incorpore uma estrutura demográfica tratável dentro de um modelo DSGE Novo Keynesiano padrão. Para tanto, nos baseamos no modelo de gerações sobrepostas. Esta estrutura dá origem a duas variáveis demográficas adicionais além da taxa de crescimento dos agentes recém-nascidos, a saber, as probabilidades de saída associadas aos dois estados, que podem ser calibradas para coincidir com as durações médias de idade ativa e reforma.

Esta característica é a chave para manter pequeno o espaço de estados do modelo, de forma que existam relações agregadas de consumo e poupança de forma fechada, apesar da heterogeneidade dos agentes no nível micro. Combinamos essa estrutura tratável com características do lado da oferta neo-keynesianas, caracterizadas por acumulação de capital, competição imperfeita no setor de bens intermediários e rigidez de preços nominais, conforme Calvo (1983). Essas características dão origem a uma curva de Phillips New Keynesiana modificada, o que implica que a dinâmica de curto prazo do framework proposto pode ser comparada com a do modelo padrão. Sob a suposição especial de que os trabalhadores têm uma vida infinita, essa dinâmica se torna idêntica à do modelo padrão. Na ausência dessa suposição especial, entretanto, a dinâmica neo-keynesiana se torna mais rica, porque o consumo atual e o futuro esperado dependem de fatores demográficos e dos efeitos do ciclo de vida individual.

Como toda a trajetória do consumo é importante para os valores atuais e futuros do produto e da inflação, isso leva a uma modificação da dinâmica de curto prazo que é relevante para os efeitos da política monetária. Para caracterizar essa dinâmica, comparamos equilíbrios sob preços flexíveis e rígidos. Essa distinção é importante para a resposta de variáveis reais e nominais e facilita uma caracterização significativa do comportamento de curto prazo da economia em resposta a choques demográficos.


Ancorar a economia ao longo do tempo em torno dos níveis-alvo para a taxa de inflação e o razão da dívida pública. Refletindo a estrutura de gerações sobreposição, a dinâmica do modelo é criticamente afetada pela política fiscal (que é, por construção, não neutra) e, em particular, pelo desenho do sistema previdenciário, que facilita as transferências intergeracionais entre trabalhadores e aposentados e determina a força dos efeitos do ciclo de vida individual.

Utilizamos o nosso modelo, calibrado para dados da área do euro, para examinar as implicações macroeconómicas selecionadas das alterações demográficas de uma perspectiva positiva. Em particular, nos concentramos nos determinantes da taxa de juros real de equilíbrio. Nossa análise prossegue em duas etapas.

Primeiro, investigamos qualitativamente as implicações de longo prazo de uma diminuição no crescimento populacional e um aumento na expectativa de vida. Para esse fim, abstraímos da dinâmica inflacionária (e de qualquer outra) de curto prazo e relatamos os resultados estáticos comparativos da versão de preço flexível de nossa estrutura. Nossa análise revela que o efeito de longo prazo sobre a taxa de juros real de equilíbrio depende criticamente de
pressupostos relativos ao curso futuro do sistema de pensões PAYGO assumido. Para ilustração, distinguimos entre dois tipos de cenários em que o aumento da taxa de dependência dos idosos leva ou não a mudanças na taxa de reposição (definida como a razão entre benefícios de aposentadoria e salários individuais). Para o primeiro tipo de cenário, a taxa de reposição diminui endogenamente, de modo que a relação benefício-produto agregado permanece inalterada. Essa suposição equivale a um fortalecimento dos elementos de financiamento privado, porque introduz um teto para a redistribuição financiada por impostos entre trabalhadores e aposentados. Para o segundo tipo de cenário, a taxa de reposição permanece constante, levando a um aumento na relação benefício-produto agregado. Esta suposição equivale a um cenário de "nenhuma reforma" que extrapola o sistema previdenciário existente para o futuro, levando a uma maior carga tributária sobre os trabalhadores. A principal conclusão é que, em qualquer um dos cenários, a redução no crescimento populacional e o aumento na expectativa de vida são duas forças independentes que contribuem para a queda na taxa de juros real de equilíbrio. No entanto, a magnitude prevista desse declínio difere entre os cenários devido aos incentivos distintos para a poupança individual. Em particular, para o primeiro tipo de cenário, o declínio é mais pronunciado devido às poupanças adicionais realizadas pelos trabalhadores na expectativa de rendimentos de pensões mais baixos no futuro.

A principal descoberta é que a desaceleração projetada no crescimento populacional e o aumento na longevidade contribuem lentamente ao longo do tempo para uma queda na taxa de juros de equilíbrio.
%
%
\section{\citet{Gagnon:2016}: Understanding the New Normal: The Role of Demographics}
Este artigo busca entender o quanto da nova normalidade pode ser explicada por fatores demográficos nos Estados Unidos.

Neste artigo, investigamos até que ponto as mudanças demográficas, especialmente aquelas relacionadas ao baby boom, podem explicar os níveis atualmente baixos das taxas de juros reais e do crescimento do PIB. Construímos um modelo de geração sobreposta (OG) que é consistente com as mudanças observadas e projetadas na fecundidade, oferta de trabalho, expectativa de vida, composição familiar e migração internacional. O modelo nos permite explorar até que ponto as mudanças demográficas, por si só, podem explicar o momento e a magnitude dos movimentos nas taxas de juros reais e no crescimento real do PIB durante o período pós-guerra.
e além. Como estamos interessados em tendências de longo prazo, não tentamos modelar a variação do ciclo de negócios, abstrair da rigidez nominal e postular que os mercados de fatores são perfeitamente competitivos. Também não assumimos nenhum motivo de herança intencional para nossas famílias; sem esses motivos, o tempo de vida finito das famílias permite que as transições demográficas influenciem as taxas de juros reais. Em contraste, no modelo padrão de ciclo de negócios real (RBC) em que as famílias têm horizontes de planejamento infinitos, as transições demográficas podem não ter esse efeito sobre as taxas de juros reais.

Descobrimos que os fatores demográficos por si só podem ser responsáveis por um declínio de 5/4 pontos percentuais na taxa de juros real de equilíbrio no modelo desde 1980 - muito, se não todo, do declínio permanente nas taxas de juros reais ao longo desse período, de acordo com alguns estimativas de séries temporais. Curiosamente, o modelo também implica que essas quedas foram mais pronunciadas desde o início de 2000, de modo que as pressões para baixo nas taxas de juros e no crescimento do PIB devido à demografia poderiam ser facilmente mal interpretadas como influências persistentes, mas, em última análise, temporárias da crise financeira global. Quando enriquecemos nosso modelo com crescimento da produtividade total dos fatores (PTF) consistente com a tendência histórica observada, os declínios previstos no crescimento do PIB e na taxa real de equilíbrio entre 1980 e o presente são essencialmente inalterados, mas o ritmo no qual a economia está em transição para o novo normal na última década é mais dramático.

Uma característica notável de nosso modelo é que os efeitos macroeconômicos da transição demográfica são amplamente previsíveis. Ao fazer isso, usamos o modelo para mostrar que as mudanças demográficas mais consequentes para as taxas de juros e o crescimento do PIB hoje ocorreram antes da década de 1980. Se as variáveis demográficas tivessem se mantido constantes em seus níveis de 1960, tanto as taxas de juros reais quanto o crescimento do PIB hoje seriam semelhantes aos seus níveis em 1980. Em nítido contraste, se as variáveis demográficas tivessem se mantido constantes em seus níveis de 1980, os declínios na taxa real de equilíbrio e o crescimento do PIB real teria sido em grande parte conforme previsto na calibração de base das variáveis demográficas.

Neste artigo, usamos um modelo OG calibrado com uma estrutura demográfica rica para investigar até que ponto as mudanças passadas e projetadas na composição familiar dos EUA, expectativa de vida e oferta de trabalho tiveram influência nas taxas de juros reais, crescimento do PIB e outras variáveis. Descobrimos que esses fatores demográficos sozinhos são responsáveis por um declínio de 5/4 pontos percentuais na taxa natural de juros e no crescimento real do PIB desde 1980 - uma magnitude que argumentamos estar de acordo com as estimativas empíricas. Além disso, o modelo é consistente com a taxa de investimento agregado tendo desacelerado consideravelmente ao longo desse período. Olhando para o futuro, o modelo sugere que as baixas taxas de juros, o baixo crescimento do produto e as baixas taxas de investimento vieram para ficar, sugerindo que a economia dos EUA entrou em uma nova normalidade.

A rapidez com que nosso modelo de economia está fazendo a transição para um novo normal sugere um risco de que os efeitos permanentes dos fatores demográficos possam ser mal interpretados como uma pressão descendente persistente, mas transitória, sobre a taxa natural de juros e a poupança líquida decorrente da crise financeira global. No futuro, nossos resultados têm implicações para avaliar a sustentabilidade fiscal. Além disso, a persistência de uma taxa de juros real de equilíbrio baixa significa que o escopo de usar a política monetária convencional para estimular a economia durante desacelerações cíclicas típicas será mais limitado do que no passado para uma determinada meta de inflação.
%
%
\section{\citet{Papetti:2020}: Demographics and the Natural Real Interest Rate: historical and projected paths for the euro area}
Eu considero um modelo OLG que abstrai de fricções nominais, imperfeições de mercado e choques diferentes daqueles implicados pela demografia. Nesse modelo, como é padrão, a taxa de juros natural é a taxa real de retorno do capital (líquida de depreciação) que permite que a oferta de poupança (das famílias) atenda à demanda de capital (das empresas). Em todo o artigo, ela é chamada abreviadamente de taxa de juros real.

Encontro uma representação agregada que lança luz sobre os canais por meio dos quais as mudanças demográficas podem afetar a taxa de juros real e forneço estimativas quantitativas. A variável principal acaba sendo a taxa de crescimento da relação trabalho-população efetiva, ou seja, a relação entre o número de pessoas em idade ativa avaliada de acordo com a produtividade dependente da idade sobre o número de pessoas no total população.

Calibro o modelo para a área do euro usando dados demográficos. Esses exercícios mostram que o efeito de amortecimento do envelhecimento poderia ser mitigado não apenas por uma maior substituibilidade entre trabalho e capital e maior elasticidade de substituição intertemporal no consumo, mas também por reformas que visam particularmente aumentar a produtividade relativa das coortes mais velhas e a taxa de participação. Um aumento da idade de aposentadoria sem nenhuma outra reforma de apoio leva a uma trajetória mais elevada da taxa de juros real, mas apenas em uma extensão limitada e quantitativamente insignificante.

MODELO DE ECONOMIA ABERTA.

Por meio de um modelo agregado, que se aproxima do caminho de solução de um modelo de geração sobreposta (OLG), este trabalho conclui que a mudança demográfica tem um impacto significativo na taxa de juro real natural para a área do euro: uma pressão ascendente nos anos 70 e 80 e uma pressão descendente prolongada que se estende pelo menos até 2030 conforme o processo de envelhecimento se desenvolve. O modelo prevê na linha de base uma diminuição da taxa de juros real natural de cerca de 1 ponto percentual de 1990 a 2030 (aproximadamente o pico ao vale na simulação) e o intervalo de estimativas está entre -1,7 e -0,4 pontos percentuais de acordo a um conjunto de especificações de sensibilidade. As estimativas sugerem que o impacto descendente do envelhecimento poderia ser mitigado não apenas por uma maior substituibilidade na produção entre trabalho e capital
e maior elasticidade intertemporal de substituição no consumo, mas também por reformas que visam principalmente aumentar a produtividade relativa das coortes mais velhas e a taxa de participação. O aumento da idade de aposentadoria per se tem apenas um efeito atenuante limitado. Existem dois fatores que explicam por que o envelhecimento tem um impacto descendente sobre a taxa de juros real natural: a mão-de-obra como insumo de produção torna-se mais escassa e os indivíduos aumentam sua disposição de economizar em antecipação a uma expectativa de vida mais longa. Ambos drivers são considerados responsáveis por
igualmente para explicar a tendência de queda da taxa de juros real natural ao longo do horizonte projetado. O fato de a taxa de poupança diminuir à medida que aumenta a fração de pessoas que se aposentam tem um efeito atenuante, mas nunca forte o suficiente para contrabalançar os dois fatores de amortecimento. A chave para entender o impacto do envelhecimento sobre a taxa de juros real natural é a evolução da taxa de crescimento da relação trabalho-população efetiva, que é a relação entre o número de pessoas em idade ativa avaliada em função da produtividade dependente da idade sobre o número de pessoas em toda a população.
%
%
\section{\citet{Eggertsson:2019}: A Model of Secular Stagnation: Theory and Quantitative Evaluation}
Este artigo contempla a possibilidade de que a queda nas taxas de juros observada nos últimos 25 anos no mundo industrializado represente uma mudança permanente - uma “nova normalidade”. Propomos um modelo que permite episódios de ZLB de duração arbitrária devido a uma queda persistente nas taxas de juros impulsionada por forças seculares lentas que dificilmente se reverterão. Também realizamos uma avaliação quantitativa para determinar a plausibilidade de um declínio na taxa de juros natural (pleno emprego) para valores negativos usando um modelo de ciclo de vida quantitativo calibrado para dados atuais dos EUA.

Essa ideia foi ressuscitada recentemente por Lawrence Summers, que argumentou que se deve pensar na estagnação secular como a hipótese de que a taxa natural de juros (a taxa de juros real de equilíbrio consistente com o produto no potencial) é permanentemente negativa (ver Summers 2013, 2014). Formalizamos essa ideia construindo uma série de modelos analíticos e quantitativos de geração sobreposta (OLG) de vários graus de complexidade nos quais a taxa de juros real em estado estacionário e de pleno emprego é permanentemente negativa. Isso leva, sob certas condições, a um ZLB cronicamente vinculativo, crescimento abaixo da média e inflação abaixo da meta; definimos essas características como uma estagnação secular.

O aumento da meta de inflação pode ser uma opção para acomodar uma taxa de juros natural negativa. Essa política, no entanto, apresenta duas desvantagens principais. Primeiro, um aumento na meta de inflação deve ser suficientemente grande. Em segundo lugar, mesmo com um aumento grande o suficiente na meta de inflação, o equilíbrio de estagnação secular não é eliminado.
Assim, o aumento da meta de inflação permite um melhor resultado, mas não o garante, porque não pode excluir um equilíbrio de estagnação secular.

A política fiscal é mais eficaz para lidar com os problemas levantados pela estagnação secular, e uma expansão fiscal agressiva o suficiente não sofre com a multiplicidade de estados estacionários - ela elimina completamente o estado estacionário de estagnação secular. No entanto, os efeitos da política fiscal são mais sutis do que no tratamento neo-keynesiano padrão do ZLB. Os aumentos nos gastos do governo podem gerar multiplicadores zero ou negativos em nosso modelo, dependendo da distribuição dos impostos entre as gerações. A chave para uma política fiscal bem-sucedida é que ela deve reduzir o excesso de oferta de poupança e aumentar a taxa natural de juros. A política fiscal que, em vez disso, aumenta a poupança desejada, por exemplo, reduzindo a renda disponível futura por meio de aumentos de impostos, pode exacerbar uma estagnação secular. No modelo NK padrão, os multiplicadores de gastos do governo estão acima de um, independentemente do mecanismo de financiamento, e podem ser ilimitadamente grandes por motivos relacionados ao  forward guidance puzzle. Esses mecanismos são consideravelmente atenuados em nosso framework.

Além de permitir a possibilidade de períodos arbitrariamente longos de taxas de juros negativas, nosso modelo considera e pode quantificar uma série de novas forças que afetam a taxa natural de juros. Essas forças entram naturalmente em ação em nossa análise, uma vez que abandonamos a estrutura do agente representativo do modelo NK padrão. Essencialmente, qualquer força que altere a oferta relativa de poupança e investimento pode ter um efeito sobre a taxa de juros. Mostramos como uma desaceleração no crescimento populacional ou um aumento na expectativa de vida pressiona para baixo a taxa natural. O aumento da desigualdade de renda ou a queda no preço relativo dos bens de investimento também podem reduzir a taxa natural. Uma desaceleração no crescimento da produtividade também pode desempenhar um papel importante.

Enquanto a primeira contribuição principal deste artigo é uma estrutura analítica que apresenta os ingredientes teóricos necessários para caracterizar a estagnação secular, a segunda contribuição principal é a construção de um modelo de ciclo de vida de média escala para explorar se as taxas de juros naturais persistentemente negativas são quantitativamente realistas. Construímos um modelo OLG de 56 períodos com capital e o calibramos para corresponder à economia dos EUA em 2015, assumindo que o hiato do produto naquele momento é (i) zero, ou (ii) corresponde ao desvio do produto em relação ao seu período anterior Tendência de 2008 - duas referências naturais (e extremas).

Nosso modelo quantitativo é capaz de gerar uma taxa de juros natural permanentemente negativa usando parâmetros que são padrão na literatura macro e correspondem a momentos-chave dos dados dos EUA. Os principais motores das taxas de juros naturais negativas são o envelhecimento da população, baixa fertilidade e crescimento lento da produtividade. Embora essa tendência possa se reverter, se as projeções atuais para fertilidade e produtividade se mantiverem, nossa análise sugere que a taxa natural de juros será baixa ou negativa em um futuro próximo. Embora o crescimento da produtividade tenha experimentado períodos inesperados de aceleração e desaceleração desde a década de 1970, é improvável que os fatores demográficos responsáveis por uma baixa taxa de juros natural diminuam.

Usamos nosso modelo para entendxer o declínio nas taxas de juros visto nos dados. Levamos nossa calibração de 2015 e revertemos os fatores demográficos e de produtividade observáveis aos seus valores de 1970. Ao longo desse período, nosso modelo gera uma redução de 4,02\% na taxa de juros real de 1970 a 2015, que corresponde à redução real de 4,09\% observada na taxa real de fundos federais naquele período. As reduções na fertilidade, mortalidade e a taxa de crescimento da produtividade desempenham o papel principal; cada um sozinho pode ser responsável por uma queda na taxa de juros real de -1,84\%, -1,92\% e -1,90\%, respectivamente. O principal fator que tende a contrabalançar essas forças é o aumento da dívida pública, que representa um aumento de 2,11\% na taxa de juros real. Mudanças na participação do trabalho, no preço relativo dos bens de investimento e na variação na capacidade de endividamento do consumidor desempenham um papel quantitativamente menor (as duas primeiras diminuem as taxas em -0,50 por cento e -0,44 por cento, enquanto o último aumenta as taxas em 0,13 por cento).

Este artigo constrói uma teoria quantitativa de taxas de juros negativas e estagnação secular. Mostramos como uma baixa taxa de juros natural pode levar a uma estagnação secular: uma queda persistente do produto, inflação abaixo da meta e um ZLB cronicamente vinculado. Nossa mensagem não é que o ZLB será vinculado para sempre com certeza. Um mundo de taxas naturais baixas admite ciclos de negócios em que a taxa de curto prazo ainda pode ser temporariamente positiva. Em vez disso, é um mundo caracterizado por uma “nova normalidade”, em que as taxas de juros reais precisam, em média, ser negativas para alcançar o pleno emprego.

Medidas que poderiam eliminar a estagnação secular por meio de políticas adequadas. Um sério desafio, no entanto, é que nossas recomendações de política advogam em favor de políticas que foram consideradas vícios em vez de virtudes na teoria macroeconômica: uma meta de inflação mais alta, aumentos persistentes na relação dívida / PIB, ou até mesmo PAYGO  para o Seguro Social.