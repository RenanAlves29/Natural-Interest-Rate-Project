\chapter{Textos da Introdução}
%
%
\section{\citet{Summers:2014}: Reflections on the new secular stagnation hypothesis. In: Secular Stagnation: Facts, Causes and Cures} 

Depois de observar a aparente dificuldade que as economias industrializadas estão tendo em alcançar um crescimento financeiramente estável com pleno emprego, explico por que uma queda na taxa de juros real de pleno emprego (FERIR) aliada a uma baixa inflação poderia impedir indefinidamente a obtenção do pleno emprego. Eu defendo que mesmo se fosse possível para o FERIR ser alcançado, isso poderia envolver uma instabilidade financeira substancial. Tendo argumentado que um declínio no FERIR explicaria muito do que observamos, a seguir acrescento uma variedade de fatores que sugerem que o FERIR diminuiu substancialmente nas últimas décadas no mundo industrial.

Já enfatizei que a flexibilidade de salários e preços pode exacerbar o problema. Quanto mais flexíveis os salários e preços, mais se espera que caiam durante uma desaceleração da produção, levando a um aumento nas taxas de juros reais. De fato, existe a possibilidade de desestabilizar a deflação com a queda dos preços, levando a taxas de juros reais mais altas, levando a maiores quebras de produto, levando a uma queda mais rápida dos preços e daí em um ciclo vicioso. (QUE DOIDO ISSO!!!)

Sugira que os níveis de FERIR podem ter diminuído substancialmente. Esses incluem:
População mais lenta e possivelmente crescimento tecnológico significam uma redução na demanda por novos bens de capital para equipar trabalhadores novos ou mais produtivos.

Bens de capital com preços mais baixos significam que um determinado nível de poupança pode comprar muito mais capital do que antes.

O aumento da desigualdade opera para aumentar a parcela da renda destinada àqueles com menor propensão a gastar.

O aumento da fricção na intermediação financeira, associado a uma maior aversão ao risco na esteira da crise financeira e ao aumento da carga regulatória, opera para aumentar a diferença entre as taxas líquidas seguras e as taxas cobradas dos tomadores.

Um desejo crescente por parte dos bancos centrais e governos de acumular reservas, juntamente com estratégias de investimento conservadoras, opera para aumentar a demanda por ativos seguros, reduzindo as taxas de juros seguras.

Desinflação contínua, o que significa que, a qualquer taxa de juros real, as taxas de juros reais após impostos são mais altas.

Há um trabalho importante a ser feito elucidando a ideia de estagnação secular em um contexto de economia aberta. A melhor maneira de pensar sobre a análise aqui é tratá-la como referindo-se à economia agregada do mundo industrial onde - por causa da mobilidade de capital - as taxas de juros reais tendem a convergir (embora não imediatamente devido à possibilidade de movimentos esperados nas taxas de câmbio reais ) Se o FERIR para as economias industrializadas fosse baixo o suficiente, poder-se-ia esperar saídas de capital para os mercados emergentes, que estariam associadas a taxas de câmbio reais em declínio para os países industrializados, maior competitividade e aumento da demanda de exportação. A dificuldade é que isso é algo que os mercados emergentes aceitarão apenas até certo ponto. É provável que sua resposta seja resistência aos influxos de capital ou esforços para administrar os valores das moedas para manter a competitividade. Em ambos os casos, o resultado será mais pressão baixista sobre as taxas de juros nos países industrializados.

Em termos gerais, na medida em que a estagnação secular é um problema, existem duas estratégias possíveis para lidar com seus impactos perniciosos.

O primeiro é encontrar formas de reduzir ainda mais as taxas de juros reais

Isso pode incluir operar com uma meta de taxa de inflação mais alta, de modo que uma taxa nominal zero corresponda a uma taxa real mais baixa. Ou pode incluir encontrar formas, como quantitative easing, que operem para reduzir o crédito ou prêmios de prazo. É claro que essas estratégias têm a dificuldade de que, mesmo que aumentem o nível de produção, também podem aumentar os riscos de estabilidade financeira, que por sua vez podem ter consequências sobre a produção.

A alternativa é aumentar a demanda aumentando o investimento e reduzindo a poupança.

Isso opera para aumentar o FERIR e, assim, promover a estabilidade financeira, bem como o aumento da produção e do emprego. Como pode ser isto alcançado? As estratégias apropriadas variam de país para país e de situação para situação. Mas eles devem incluir aumento do investimento público, redução das barreiras estruturais ao investimento privado e medidas para promover a confiança empresarial, um compromisso de manter proteções sociais básicas de modo a manter o poder de compra e medidas para reduzir a desigualdade e, assim, redistribuir a renda para aqueles com uma maior propensão gastar.
%
%
\section{\citet{Gordon:2015}: Secular Stagnation: A Supply-Side View}
O lado da oferta da estagnação secular refere-se ao crescimento potencial do PIB real, a taxa de crescimento do produto consistente com a inflação não acelerada.

A estagnação secular na forma de crescimento lento do produto potencial na última meia década reflete a lentidão do crescimento da produtividade do trabalho e das horas agregadas de trabalho, e o crescimento lento neste último se deve tanto à desaceleração do crescimento populacional quanto ao declínio na Taxa de Participação da Força de Trabalho (LFPR). Como o comportamento do LFPR tem recebido ampla atenção de pesquisa recentemente, este artigo enfoca as fontes de crescimento lento da produtividade.

O documento fornece três argumentos separados para explicar o lento crescimento da produtividade na última década. A primeira é que as mudanças fundamentais nos métodos de negócios se concentraram na era dot.com de rápido crescimento da produtividade e, uma vez que novos equipamentos foram instalados e novas práticas de negócios foram adotadas, o impacto da revolução das TIC no crescimento da produtividade começou a ter retornos decrescentes. Um segundo argumento aponta para as medidas de desempenho econômico que tiveram o mesmo timing, com pico no final da década de 1990 e caindo para níveis baixos nos últimos anos, incluindo o crescimento da capacidade de manufatura, a relação entre o investimento líquido e o estoque de capital, a taxa de declínio no deflator de preços de TIC e a velocidade de melhoria da tecnologia de microchip. Outra medida da queda do desempenho econômico inclui a taxa de abertura de novas empresas.

O crescimento mais lento do produto potencial do lado da oferta, proveniente não apenas do crescimento lento da produtividade, mas do crescimento mais lento da população e do declínio da participação da força de trabalho, reduz a necessidade de formação de capital, e isso, por sua vez, subtrai da demanda agregada e reforça o declínio da produtividade crescimento. No final das contas, a estagnação secular não é apenas sobre oferta ou demanda, mas também sobre a interação entre demanda e oferta.

Outro artigo do Summers 2014:  U.S. Economic Prospects: Secular Stagnation, Hysteresis, and the Zero Lower Bound

Em minhas observações hoje, quero abordar essas questões - estagnação secular, a ideia de que a economia se reequilibra; histerese, a sombra projetada sobre a atividade econômica por desenvolvimentos cíclicos adversos; e a importância do limite inferior zero para a eficácia relativa das políticas monetária e fiscal.

Vou argumentar três proposições. Em primeiro lugar, como os Estados Unidos e outras economias industriais estão configurados atualmente, a obtenção simultânea de crescimento adequado, utilização da capacidade e estabilidade financeira parece cada vez mais difícil. Em segundo lugar, é provável que isso esteja relacionado a um declínio substancial no equilíbrio ou na taxa real de juros natural. Terceiro, enfrentar esses desafios requer políticas diferentes abordagens do que as representadas pela sabedoria convencional atual.

Imaginemos, como hipótese, que tenha ocorrido esse declínio na taxa real de juros de equilíbrio. O que se esperaria ver? Seria de se esperar uma dificuldade crescente, especialmente na fase de baixa do ciclo, em alcançar o pleno emprego e um forte crescimento, devido às restrições associadas ao limite inferior zero das taxas de juros. Seria de se esperar que, normalmente, as taxas de juros reais fossem menores. Com taxas de juros reais muito baixas e com inflação baixa, isso também significa taxas de juros nominais muito baixas, portanto, seria de se esperar uma crescente busca de risco por parte dos investidores.

Portanto, acho razoável sugerir que, se houvesse um declínio significativo nas taxas de juros reais de equilíbrio, poderíamos observar os tipos de sinais perturbadores que observamos. É razoável sugerir que as taxas de juros reais de equilíbrio caíram?
Eu sugeriria que é uma hipótese razoável por pelo menos seis razões, cujo impacto difere de momento a momento e provavelmente não é facilmente passível de quantificação precisa.

Primeiro, as reduções na demanda por investimento financiado por dívida. Em parte, isso é um reflexo do legado de um período de alavancagem excessiva. Em segundo lugar, é bem sabido que uma taxa decrescente de crescimento populacional significa uma taxa natural decrescente de juros. Existe a possibilidade, sobre a qual não me defendo, de que o ritmo do progresso tecnológico também tenha diminuído, funcionando em uma direção semelhante. Terceiro, as mudanças na distribuição da renda, tanto entre a renda do trabalho e a renda do capital quanto entre aqueles com mais e aqueles com menos riqueza, têm operado para aumentar a propensão a poupar, assim como os aumentos nos lucros retidos pelas empresas.

Eu diria primeiro que há um desafio contínuo de como alcançar crescimento com estabilidade financeira. Em segundo lugar, isso pode ser o que você esperaria se houvesse um declínio substancial nas taxas de juros reais naturais. E em terceiro lugar, lidar com esses desafios requer uma consideração cuidadosa sobre quais abordagens de políticas devem ser seguido.
%
%
\section{\citet{Bernhardsen:2007}: The neutral real interest rate}
A taxa de juros é o instrumento de política monetária mais importante. Pode ser definido de forma que a política monetária seja expansionista, contracionista ou neutra. O conceito de “taxa de juros real neutra” geralmente está associado ao nível da taxa de juros real, o que implica que a política monetária não é expansionista nem contracionista. Se o banco central pretende estimular a atividade econômica, a taxa de juros deve ser fixada de forma que a taxa real de juros seja inferior à taxa neutra. Se o banco central tem como objetivo amortecer a atividade, a taxa de juros deve ser fixada de forma que a taxa de juros real seja superior à taxa neutra.

A taxa de juros real neutra é um conceito importante, no entanto, para avaliar a orientação da política monetária. Os bancos centrais devem ter uma percepção de como a política monetária expansionista ou contracionista é. Isso requer uma avaliação do nível da taxa de juros real neutra.

Existem vários conceitos de taxas de juros reais. É particularmente importante distinguir entre a taxa de juros real de equilíbrio de longo prazo, a taxa de juros real neutra e a taxa de juros real real. A taxa de juros real de equilíbrio de longo prazo é determinada por fundamentos econômicos, como potencial de crescimento e comportamento da poupança privada. Além disso, a taxa de juros real neutra é determinada por vários distúrbios que afetam a oferta e a demanda da economia no médio prazo. A taxa de juros real neutra pode se desviar da taxa de juros real de equilíbrio de longo prazo, mas se moverá em direção a ela ao longo do tempo. A taxa de juros real real é amplamente determinada pelo nível da taxa de política oficial do banco central e, portanto, depende dos objetivos da política monetária e dos distúrbios aos quais a economia está exposta. A taxa de juros real real pode, portanto, diferir dos juros reais neutros taxa por períodos de tempo mais curtos ou mais longos.

Taxa de juros real de equilíbrio de longo prazo: determinada por fundamentos econômicos, como comportamento de poupança de longo prazo, produtividade e crescimento populacional.

Taxa de juros real neutra: Determinada por todas as perturbações na economia que influenciam a perspectiva de redução do hiato do produto no médio prazo. Incluem os fundamentos que determinam a taxa de juros real de equilíbrio de longo prazo, mas também distúrbios de natureza mais temporária.

Taxa de juros real real: determinada pelo desejo do banco central de conduzir uma política monetária expansionista ou contracionista. Quando ocorrem perturbações econômicas, o banco central fixa a taxa de juros real abaixo ou acima do nível neutro com o objetivo de estabilizar a economia de forma que os objetivos de política monetária sejam alcançados.

Uma pequena economia aberta é fortemente influenciada por fatores globais. Um possível ponto de partida para discutir as taxas de juros em uma pequena economia aberta é a paridade de taxa de juros não coberta ajustada ao risco.

Quando o prêmio de risco é zero, a paridade da taxa de juros não coberta é mantida. O retorno esperado do investimento global (medido em moeda nacional) é então igual ao retorno do investimento no país de origem. Se o retorno esperado sobre o investimento global for diferente do retorno sobre o investimento doméstico, os investidores mudarão para investimentos que gerem os maiores retornos. Suponha, por exemplo, que a taxa de juros global caia. Os títulos de renda fixa domésticos serão, então, mais atraentes para investidores nacionais e estrangeiros. Exija por eles aumentará, levando à redução das taxas de juros internas e à valorização da moeda nacional.

Assim como as taxas de juros nominais globais podem influenciar as taxas de juros nominais domésticas, o comportamento global da poupança e do investimento e a taxa de juros real neutra global podem influenciar a taxa de juros real neutra em uma economia pequena e aberta. Não existe relacionamento simples
entre a taxa de juros real neutra global e a taxa de juros real neutra em uma economia pequena e aberta. A relação dependerá de como funcionam as economias e das perturbações a que estão expostas. Perturbações globais podem ter efeito cascata para a demanda e o lado da oferta de uma economia pequena e aberta e, portanto, contribuem para que o produto se desvie do produto potencial. Perturbações que surgem em uma economia pequena e aberta normalmente não afetarão o desenvolvimento econômico no resto do mundo. Uma análise detalhada dessas relações exigem um modelo da economia global e da economia doméstica. Limitar-nos-emos aqui a apontar alguns mecanismos que podem contribuir para a compreensão de como a taxa de juros real neutra em uma economia pequena e aberta pode ser influenciada por fatores globais.
%
%
\section{\citet{Caballero:2017}: The Safety Trap}